% Options for packages loaded elsewhere
\PassOptionsToPackage{unicode}{hyperref}
\PassOptionsToPackage{hyphens}{url}
%
\documentclass[
  letterpaper,
]{book}

\usepackage{amsmath,amssymb}
\usepackage{iftex}
\ifPDFTeX
  \usepackage[T1]{fontenc}
  \usepackage[utf8]{inputenc}
  \usepackage{textcomp} % provide euro and other symbols
\else % if luatex or xetex
  \usepackage{unicode-math}
  \defaultfontfeatures{Scale=MatchLowercase}
  \defaultfontfeatures[\rmfamily]{Ligatures=TeX,Scale=1}
\fi
\usepackage{lmodern}
\ifPDFTeX\else  
    % xetex/luatex font selection
  \setmainfont[]{Mont}
\fi
% Use upquote if available, for straight quotes in verbatim environments
\IfFileExists{upquote.sty}{\usepackage{upquote}}{}
\IfFileExists{microtype.sty}{% use microtype if available
  \usepackage[]{microtype}
  \UseMicrotypeSet[protrusion]{basicmath} % disable protrusion for tt fonts
}{}
\makeatletter
\@ifundefined{KOMAClassName}{% if non-KOMA class
  \IfFileExists{parskip.sty}{%
    \usepackage{parskip}
  }{% else
    \setlength{\parindent}{0pt}
    \setlength{\parskip}{6pt plus 2pt minus 1pt}}
}{% if KOMA class
  \KOMAoptions{parskip=half}}
\makeatother
\usepackage{xcolor}
\setlength{\emergencystretch}{3em} % prevent overfull lines
\setcounter{secnumdepth}{5}
% Make \paragraph and \subparagraph free-standing
\ifx\paragraph\undefined\else
  \let\oldparagraph\paragraph
  \renewcommand{\paragraph}[1]{\oldparagraph{#1}\mbox{}}
\fi
\ifx\subparagraph\undefined\else
  \let\oldsubparagraph\subparagraph
  \renewcommand{\subparagraph}[1]{\oldsubparagraph{#1}\mbox{}}
\fi


\providecommand{\tightlist}{%
  \setlength{\itemsep}{0pt}\setlength{\parskip}{0pt}}\usepackage{longtable,booktabs,array}
\usepackage{calc} % for calculating minipage widths
% Correct order of tables after \paragraph or \subparagraph
\usepackage{etoolbox}
\makeatletter
\patchcmd\longtable{\par}{\if@noskipsec\mbox{}\fi\par}{}{}
\makeatother
% Allow footnotes in longtable head/foot
\IfFileExists{footnotehyper.sty}{\usepackage{footnotehyper}}{\usepackage{footnote}}
\makesavenoteenv{longtable}
\usepackage{graphicx}
\makeatletter
\def\maxwidth{\ifdim\Gin@nat@width>\linewidth\linewidth\else\Gin@nat@width\fi}
\def\maxheight{\ifdim\Gin@nat@height>\textheight\textheight\else\Gin@nat@height\fi}
\makeatother
% Scale images if necessary, so that they will not overflow the page
% margins by default, and it is still possible to overwrite the defaults
% using explicit options in \includegraphics[width, height, ...]{}
\setkeys{Gin}{width=\maxwidth,height=\maxheight,keepaspectratio}
% Set default figure placement to htbp
\makeatletter
\def\fps@figure{htbp}
\makeatother

\makeatletter
\makeatother
\makeatletter
\@ifpackageloaded{bookmark}{}{\usepackage{bookmark}}
\makeatother
\makeatletter
\@ifpackageloaded{caption}{}{\usepackage{caption}}
\AtBeginDocument{%
\ifdefined\contentsname
  \renewcommand*\contentsname{Содержание}
\else
  \newcommand\contentsname{Содержание}
\fi
\ifdefined\listfigurename
  \renewcommand*\listfigurename{Список Иллюстраций}
\else
  \newcommand\listfigurename{Список Иллюстраций}
\fi
\ifdefined\listtablename
  \renewcommand*\listtablename{Список Таблиц}
\else
  \newcommand\listtablename{Список Таблиц}
\fi
\ifdefined\figurename
  \renewcommand*\figurename{Рисунок}
\else
  \newcommand\figurename{Рисунок}
\fi
\ifdefined\tablename
  \renewcommand*\tablename{Таблица}
\else
  \newcommand\tablename{Таблица}
\fi
}
\@ifpackageloaded{float}{}{\usepackage{float}}
\floatstyle{ruled}
\@ifundefined{c@chapter}{\newfloat{codelisting}{h}{lop}}{\newfloat{codelisting}{h}{lop}[chapter]}
\floatname{codelisting}{Список}
\newcommand*\listoflistings{\listof{codelisting}{Список Каталогов}}
\makeatother
\makeatletter
\@ifpackageloaded{caption}{}{\usepackage{caption}}
\@ifpackageloaded{subcaption}{}{\usepackage{subcaption}}
\makeatother
\makeatletter
\@ifpackageloaded{tcolorbox}{}{\usepackage[skins,breakable]{tcolorbox}}
\makeatother
\makeatletter
\@ifundefined{shadecolor}{\definecolor{shadecolor}{rgb}{.97, .97, .97}}
\makeatother
\makeatletter
\makeatother
\makeatletter
\makeatother
\ifLuaTeX
\usepackage[bidi=basic]{babel}
\else
\usepackage[bidi=default]{babel}
\fi
\babelprovide[main,import]{russian}
% get rid of language-specific shorthands (see #6817):
\let\LanguageShortHands\languageshorthands
\def\languageshorthands#1{}
\ifLuaTeX
  \usepackage{selnolig}  % disable illegal ligatures
\fi
\IfFileExists{bookmark.sty}{\usepackage{bookmark}}{\usepackage{hyperref}}
\IfFileExists{xurl.sty}{\usepackage{xurl}}{} % add URL line breaks if available
\urlstyle{same} % disable monospaced font for URLs
\hypersetup{
  pdftitle={Greek-Slavonic Parallel Corpus},
  pdfauthor={Лев Шадрин, Анастасия Дрожжина},
  pdflang={ru},
  hidelinks,
  pdfcreator={LaTeX via pandoc}}

\title{Greek-Slavonic Parallel Corpus}
\author{Лев Шадрин, Анастасия Дрожжина}
\date{2023-04-07}

\begin{document}
\frontmatter
\maketitle
\ifdefined\Shaded\renewenvironment{Shaded}{\begin{tcolorbox}[sharp corners, boxrule=0pt, interior hidden, enhanced, breakable, frame hidden, borderline west={3pt}{0pt}{shadecolor}]}{\end{tcolorbox}}\fi

\renewcommand*\contentsname{Содержание}
{
\setcounter{tocdepth}{2}
\tableofcontents
}
\mainmatter
\bookmarksetup{startatroot}

\hypertarget{sec-about_project}{%
\chapter{О проекте}\label{sec-about_project}}

Проект \emph{Greek-Slavonic Parallel Corpus} (GSPC) является результатом
работы студентов магистратуры ``Цифровые методы в гуманитарных науках''
НИУ ВШЭ.

Цель проекта -- создание алгоритма выравнивания богослужебных текстов
сложной структуры внутри билингвального корпуса. На данном этапе работы
для проектирования алгоритма используются тексты Цветной Триоди на
греческом и церковнославянском языках.

Работа над проектом началась в декабре 2021 года в рамках
научно-исследовательского проектного семинара программы ``Цифровые
методы в гуманитарных науках'' НИУ-ВШЭ.

\hypertarget{sec-authors}{%
\section{Участники}\label{sec-authors}}

\begin{itemize}
\item
  Шадрин Лев (добавить инфо о нас?)
\item
  Дрожжина Анастасия
\end{itemize}

\hypertarget{sec-acknowledgements}{%
\section{Незаменимые помощники проекта}\label{sec-acknowledgements}}

\begin{itemize}
\item
  игумен Пантелеимон (Королев), настоятель Данилова монастыря в
  Переславле-Залесском
\item
  Анастасия Александровна Бонч-Осмоловская
\end{itemize}

\hfill\break

\bookmarksetup{startatroot}

\hypertarget{ux446ux432ux435ux442ux43dux430ux44f-ux442ux440ux438ux43eux434ux44c}{%
\chapter{Цветная
Триодь}\label{ux446ux432ux435ux442ux43dux430ux44f-ux442ux440ux438ux43eux434ux44c}}

Цветная Триодь включает в себя богослужебные последования пасхального
цикла, то есть круг подвижных праздников от самой Пасхи до Недели Всех
святых, даты которых меняются в зависимости от дня празднования Пасхи.

Высокая значимость Цветной Триоди как компонента годового круга
церковных служений способствовала выбору этого текста в качестве
материала для нашей работы.

Греческий текст Цветной Триоди выгружен с
\href{https://glt.goarch.org/\#04}{сайта Греческой православной
архиепископии Америки}.

Церковнославянский текст Цветной Триоди выгружен с
\href{https://azbyka.ru/otechnik/Pravoslavnoe_Bogosluzhenie/triod-tsvetnaja/}{сайта
проекта «Азбука веры»}.

\hypertarget{ux441ux43fux435ux446ux438ux444ux438ux43aux430-ux442ux435ux43aux441ux442ux430-ux446ux432ux435ux442ux43dux43eux439-ux442ux440ux438ux43eux434ux438}{%
\section{\texorpdfstring{\textbf{Специфика текста Цветной
триоди}}{Специфика текста Цветной триоди}}\label{ux441ux43fux435ux446ux438ux444ux438ux43aux430-ux442ux435ux43aux441ux442ux430-ux446ux432ux435ux442ux43dux43eux439-ux442ux440ux438ux43eux434ux438}}

\hypertarget{ux442ux435ux43cux43fux43eux440ux430ux43bux44cux43dux430ux44f-ux441ux442ux440ux443ux43aux442ux443ux440ux430}{%
\subsection{Темпоральная
структура}\label{ux442ux435ux43cux43fux43eux440ux430ux43bux44cux43dux430ux44f-ux441ux442ux440ux443ux43aux442ux443ux440ux430}}

Цветная триодь имеет строгую внутреннюю временную структуру, которая
включает в себя следующие уровни:

\begin{itemize}
\item
  Неделя
\item
  День недели
\item
  Служба
\end{itemize}

---\textgreater{} На уровне темпоральной организации от элайнера
ожидается строгое выравнивание элементов в билингвальном корпусе: неделя
к неделе, день ко дню ИЛИ требуется разметка текста, которая бы
позволила работать на более мелких уровнях, а не выравнивать текст
целиком

\hypertarget{ux444ux443ux43dux43aux446ux438ux43eux43dux430ux43bux44cux43dux430ux44f-ux441ux442ux440ux443ux43aux442ux443ux440ux430}{%
\subsection{Функциональная
структура}\label{ux444ux443ux43dux43aux446ux438ux43eux43dux430ux43bux44cux43dux430ux44f-ux441ux442ux440ux443ux43aux442ux443ux440ux430}}

Внутри текста Цветной Триоди можно выделить 2 уровня:

\begin{itemize}
\item
  служебные комментариями по организации и проведению службы
\item
  тексты молитв и песнопений
\end{itemize}

---\textgreater{} Бинарная функциональная структура является одной из
причин расхождения длины богослужебных текстов для разных языков. Так, в
греческом тексте Цветной Триоди служебные комментарии представлены в
значительно сокращенном виде по сравнению со схожими элементами
церковнославянского текста.~

\hypertarget{ux43fux43eux440ux44fux434ux43eux43a-ux441ux43bux435ux434ux43eux432ux430ux43dux438ux44f-ux444ux440ux430ux433ux43cux435ux43dux442ux43eux432}{%
\subsection{Порядок следования
фрагментов}\label{ux43fux43eux440ux44fux434ux43eux43a-ux441ux43bux435ux434ux43eux432ux430ux43dux438ux44f-ux444ux440ux430ux433ux43cux435ux43dux442ux43eux432}}

Тексты песнопений на различных языках могут иметь отличный друг от друга
порядок следований, что создает перекрестную структуру внутри
билингвального корпуса и значительно усложняет задачу параллельного
выравнивания.

---\textgreater{} Разрозненный порядок следования молитв внутри
элементов службы или дня затрудняет работу классических элайнеров
(см.пункт ``Готовые решения''). Необходимо устанавливать широкое окно
для оценки близости различных компонентов

\hypertarget{ux442ux435ux43aux441ux442ux44b-ux43cux43eux43bux438ux442ux432}{%
\subsection{Тексты
молитв}\label{ux442ux435ux43aux441ux442ux44b-ux43cux43eux43bux438ux442ux432}}

В текстах Цветной Триоди на обоих языках молитвы могут быть представлены
как в сокращенном формате, так и целиком.

После текста молитвы может следовать как краткое служебное указание о
повторении молитвы. В иных случаях, текст молитвы может дублироваться
полностью в последующих абзацах, повторяясь необходимое количество раз
для исполнения во время службы.

---\textgreater{} Для решения задачи по выравниванию корпуса требуется
нормализация сводимых элементов.

\bookmarksetup{startatroot}

\hypertarget{ux43fux440ux435ux434ux432ux430ux440ux438ux442ux435ux43bux44cux43dux430ux44f-ux43eux431ux440ux430ux431ux43eux442ux43aux430-ux442ux435ux43aux441ux442ux43eux432}{%
\chapter{\texorpdfstring{\textbf{Предварительная обработка
текстов}}{Предварительная обработка текстов}}\label{ux43fux440ux435ux434ux432ux430ux440ux438ux442ux435ux43bux44cux43dux430ux44f-ux43eux431ux440ux430ux431ux43eux442ux43aux430-ux442ux435ux43aux441ux442ux43eux432}}

\hypertarget{ux440ux430ux431ux43eux442ux430-ux441-html}{%
\section{Работа с
html}\label{ux440ux430ux431ux43eux442ux430-ux441-html}}

На веб-страницах, с которых были выгружены тексты для работы, фрагменты
молитв и служебных указаний размечены при помощи параметров цвета внутри
html-тегов.

Тексты молитв выделены черным цветом, указаний - красным. Предположив,
что сохранение данной структуры (см. особенность 2 раздела ``Специфика
текста Цветной триоди (в частности) и богослужебных текстов'') поможет
улучшить результаты выравнивания корпуса, мы решили выгрузить тексты с
сохранением цветовой разметки.

\begin{itemize}
\tightlist
\item
  Церковнославянский текст имел правильную html-структуру, сведения о
  цветовой маркировки были получены через
  \href{https://github.com/Drozhzhinastya/GSPC/blob/main/texts/csl_tsvetnaya_triod/scrape.ipynb}{регулярные
  выражения}
\item
  При работе с html структурой греческого текста мы столкнулись с
  ошибками в организации страницы: многие текстовые фрагменты не
  попадали в теги цветовой разметки. Для корректной выгрузки греческого
  текста был написан код {[}ССЫЛКА{]}
\end{itemize}

Цветовая маркировка отображается в
\href{https://github.com/Drozhzhinastya/GSPC/tree/main/csv/markup}{csv}
- текст поделен между колонками \emph{black\_text} (тексты служб) и
\emph{red\_text} (комментарии к текстам), сведения о структуре
отражаются в столбце \emph{color (значения `red' и `black'
соответственно).}

\hypertarget{ux442ux435ux43cux43fux43eux440ux430ux43bux44cux43dux430ux44f-ux440ux430ux437ux43cux435ux442ux43aux430}{%
\section{Темпоральная
разметка}\label{ux442ux435ux43cux43fux43eux440ux430ux43bux44cux43dux430ux44f-ux440ux430ux437ux43cux435ux442ux43aux430}}

Темпоральная разметка осуществлена вручную, отдельно для греческого и
церковнославянского текстов, с двухуровневой структурой:

\begin{itemize}
\tightlist
\item
  Недели (1-8)
\item
  Дни недели (1\_1-1\_2-1\_3 .. 8\_6-8\_7.8\_8)
\end{itemize}

Темпоральная разметка содержится в
\href{https://github.com/Drozhzhinastya/GSPC/tree/main/csv/markup}{csv}
в столбце temp

\hypertarget{ux440ux430ux437ux43bux438ux447ux43dux44bux439-ux43fux43eux440ux44fux434ux43eux43a-ux441ux43bux435ux434ux43eux432ux430ux43dux438ux44f-ux43cux43eux43bux438ux442ux432-ux438-ux43fux435ux441ux43dux43eux43fux435ux43dux438ux439}{%
\section{Различный порядок следования молитв и
песнопений}\label{ux440ux430ux437ux43bux438ux447ux43dux44bux439-ux43fux43eux440ux44fux434ux43eux43a-ux441ux43bux435ux434ux43eux432ux430ux43dux438ux44f-ux43cux43eux43bux438ux442ux432-ux438-ux43fux435ux441ux43dux43eux43fux435ux43dux438ux439}}

Оптимальными решениями для работы с этой особенностью служебных текстов,
на наш взгляд являются инструменты, которые позволяют сопоставлять
единицы, находящиеся на большом расстоянии друг от друга в тексте (см.
text-similarity)

\hypertarget{ux440ux430ux437ux43bux438ux447ux43dux43eux435-ux43eux444ux43eux440ux43cux43bux435ux43dux438ux435-ux442ux435ux43aux441ux442ux43eux432-ux43cux43eux43bux438ux442ux432}{%
\section{Различное оформление текстов
молитв}\label{ux440ux430ux437ux43bux438ux447ux43dux43eux435-ux43eux444ux43eux440ux43cux43bux435ux43dux438ux435-ux442ux435ux43aux441ux442ux43eux432-ux43cux43eux43bux438ux442ux432}}

Развернутая структура оформления текста молитв сохраняется в csv файлах.
Однако мы предполагаем, что для оптимизации параметров длины исходных
текстов, релевантна работа с
\href{https://github.com/Drozhzhinastya/GSPC/tree/main/texts/unique_units}{уникальными
единицами текста}.

\bookmarksetup{startatroot}

\hypertarget{ux433ux43eux442ux43eux432ux44bux435-ux43fux430ux43aux435ux442ux44b-ux440ux435ux448ux435ux43dux438ux439-ux44dux43bux430ux439ux43dux435ux440ux44b}{%
\chapter{\texorpdfstring{\textbf{Готовые пакеты решений
(элайнеры)}}{Готовые пакеты решений (элайнеры)}}\label{ux433ux43eux442ux43eux432ux44bux435-ux43fux430ux43aux435ux442ux44b-ux440ux435ux448ux435ux43dux438ux439-ux44dux43bux430ux439ux43dux435ux440ux44b}}

На начальных этапах проекта предполагалось использовать элайнеры --
готовые программные решения для задачи выравнивания текстов: Lingtrain
Aligner, а также Hunalign и его пользовательскую оболочку LF-aligner.
Рассмотрим каждый из инструментов в отдельности.

\hypertarget{lf-aligner}{%
\section{LF-aligner}\label{lf-aligner}}

\emph{LF-aligner является графическим интерфейсом
\href{https://github.com/danielvarga/hunalign}{Hunalign}}. Этот
инструмент требует предварительной лемматизации (ссылка на раздел)
обрабатываемых текстов.

Алгоритм \emph{Hunalign} решает задачу выравнивания лемматизированных
текстов по предложениям или параграфам -- что соответствует элементам
молитв в нашем случае.

В \emph{Hunalign} можно дополнительно подгрузить словарь переводных
эквивалентов -- в таком случае, алгоритм будет использовать его,
комбинируя словарные данные с информацией о длине предложения по
алгоритму Гейла-Черча.

В отсутствии внешнего словаря алгоритм сперва обращается к длине
предложения, а затем автоматически генерирует словарь на основе
результатов первичного выравнивания по длине. После этого
\emph{Hunalign} выравнивает текст повторно, используя сгенерированные
словарные данные.

При работе с \emph{Hunalign} возникает проблема, связанная с отсутствием
параметров контроля окна сравнения единиц текста, из-за чего инструмент
не может справиться со сложной перекрестной структурой Цветной Триоди
(см. особенность 3 раздела Специфика текста Цветной триоди (в частности)
и богослужебных текстов).

\hypertarget{lingtrain-aligner}{%
\section{Lingtrain-aligner}\label{lingtrain-aligner}}

Среди преимуществ работы с
\href{https://github.com/averkij/lingtrain-aligner}{Lingtrain Aligner}
стоит отметить нетребовательность инструмента к предварительной
подготовке текстов, а также относительную простоту запуска алгоритмов
выравнивания.

Пользователю не требуется осуществлять токенизацию и лемматизацию
текста. Необходимо лишь определенным образом
\href{https://habr.com/ru/articles/590549/}{разметить текст} и
убедиться, что он соответствует следующим параметрам:

\begin{itemize}
\tightlist
\item
  в случае если проставлены метки для заголовков, их число должно быть
  одинаковым для обоих текстов;
\item
  в тексте отсутствуют строки, заканчивающиеся точкой и не являющиеся
  концом абзаца; в противном случае, абзацы будут сегментированы по
  точкам
\end{itemize}

Работать с подготовленными корпусами можно из python при помощи
библиотеки lingtrain-aligner.

Предварительная разметка оказывается полезной не только для выравнивания
в элайнерах, но также помогает при создании визуализации результатов --
html-книги с параллельными текстами.

Формат выдачи результатов в виде html-книги, предусмотренный в
lingtrain, не является оптимальным для целей проекта на данном этапе,
поскольку мы заинтересованы в получении словаря переводных эквивалентов
на уровне отдельных гимнов или предложений, а также в оценке качества
выравнивания.

\href{https://github.com/Drozhzhinastya/GSPC/blob/main/scripts/aligners/Lingtrain_LABse.ipynb}{Код}
для работы с текстами Цветной Триоди.

\bookmarksetup{startatroot}

\hypertarget{ux43bux435ux43cux43cux430ux442ux438ux437ux430ux446ux438ux44f}{%
\chapter{\texorpdfstring{\textbf{Лемматизация}}{Лемматизация}}\label{ux43bux435ux43cux43cux430ux442ux438ux437ux430ux446ux438ux44f}}

\hypertarget{ux434ux43bux44f-ux447ux435ux433ux43e}{%
\section{Для чего?}\label{ux434ux43bux44f-ux447ux435ux433ux43e}}

\begin{enumerate}
\def\labelenumi{\arabic{enumi}.}
\tightlist
\item
  \emph{Лемматизация как инструмент работы с элайнерами}
\end{enumerate}

При решении задачи параллельного выравнивания текстов лемматизация может
оказаться необходимой для дальнейшей обработки текста с помощью
элайнеров или пайплайна, использующего словарь переводных эквивалентов
(в нашем случае, при работе с LF-aligner).

В других ситуациях -- например, при использовании векторных моделей
наподобие Lingtrain Aligner -- необходимость в лемматизации текстов
пропадает.

\begin{enumerate}
\def\labelenumi{\arabic{enumi}.}
\tightlist
\item
  \emph{Улучшение качества лемматизации служебных текстов может являться
  отдельной исследовательской задачей, актуальной для двух направлений
  развития проекта:}
\end{enumerate}

\begin{itemize}
\tightlist
\item
  создание словаря переводных эквивалентов
\item
  оценка и улучшение качества выравнивания текстов по предложениям
  (см.пункт `Вопрос оценки качества выравнивания')
\end{itemize}

\hypertarget{ux43bux435ux43cux43cux430ux442ux438ux437ux430ux446ux438ux44f-ux442ux435ux43aux441ux442ux43eux432-ux446ux432ux435ux442ux43dux43eux439-ux442ux440ux438ux43eux434ux438}{%
\section{Лемматизация текстов Цветной
триоди}\label{ux43bux435ux43cux43cux430ux442ux438ux437ux430ux446ux438ux44f-ux442ux435ux43aux441ux442ux43eux432-ux446ux432ux435ux442ux43dux43eux439-ux442ux440ux438ux43eux434ux438}}

\hypertarget{ux446ux435ux440ux43aux43eux432ux43dux43eux441ux43bux430ux432ux44fux43dux441ux43aux438ux439-ux442ux435ux43aux441ux442}{%
\subsection{Церковнославянский
текст}\label{ux446ux435ux440ux43aux43eux432ux43dux43eux441ux43bux430ux432ux44fux43dux441ux43aux438ux439-ux442ux435ux43aux441ux442}}

В условиях отсутствия готового модуля, позволяющего провести
лемматизацию церковнославянского текста, а также учитывая описанную
специфику Цветной Триоди, было решено попробовать лемматизировать
Триодь, экспериментируя с моделями современного русского языка (Spacy,
PyMorphy2, UDPipe), так и используя модели древних языков (UDPipe).

Результаты лемматизации Spacy и PyMorphy2 показали крайне низкое
качество, и эти инструменты были исключены из списка рассматриваемых.

Для русского языка -- современного и древнего -- в UDPipe представлено 6
моделей. В таблице ниже изложены описания данных моделей:

\begin{longtable}[]{@{}
  >{\raggedright\arraybackslash}p{(\columnwidth - 2\tabcolsep) * \real{0.5000}}
  >{\raggedright\arraybackslash}p{(\columnwidth - 2\tabcolsep) * \real{0.5000}}@{}}
\toprule\noalign{}
\begin{minipage}[b]{\linewidth}\raggedright
Модель
\end{minipage} & \begin{minipage}[b]{\linewidth}\raggedright
Тексты, корпуса, трибанки модели
\end{minipage} \\
\midrule\noalign{}
\endhead
\bottomrule\noalign{}
\endlastfoot
old\_church\_slavonic-proiel-ud-2.6-200830 & тексты старославянских
памятников, представленные в корпусе PROIEL \\
old\_russian-rnc-ud-2.6-200830 & памятники древнерусской и
церковнославянской литературы, представленные в базе Национального
корпуса русского языка \\
old\_russian-torot-ud-2.6-200830 & корпус древнерусских и
старославянских текстов Torot \\
russian-syntagrus-ud-2.6-200830 & художественные тексты и новостные
издания современного русского языка аннотированного корпуса SynTagRus \\
russian-gsd-ud-2.6-200830 & конвертированный корпус Google Stanford
Dependencies \\
russian-taiga-ud-2.6-200830 & художественные, новостные, научные
датасеты, субтитры и поэзия современного русского языка, представленные
в корпусе Taiga \\
\end{longtable}

Для оценки качества моделей, каждая из них была использована при
лемматизации тестового фрагмента Триоди -- текста службы Вознесения
{[}ССЫЛКА{]}.

Наименее удовлетворительные результаты были получены при работе с
церковнославянской моделью
\emph{old\_church\_slavonic-proiel-ud-2.6-200830}. Модели, основанные на
корпусах Torot и RNC проявили себя лучше. Однако, Torot оказалась
чувствительной к знакам препинания -- при работе с этой моделью текст
нуждается в предварительном удалении пунктуации.

Ошибки лемматизации были в основном общие -- например, парсеру не
удалось правильно построить инфинитивы предиката \emph{празднуем} и
\emph{глаголет}. С \emph{глаголет} не справились и современные модели
(SynTagRus, GSD, Taiga). В целом результаты последних были довольно
ровными, хотя с некоторыми словоформами Taiga справлялась лучше, чем
SynTagRus и GSD. Например, ей удалось распарсить словоформу
\emph{вечерни (вечерня)}, с которой из других моделей справилась только
Torot.

Помимо проблемы правильного построения начальной формы, возникли ошибки,
связанные с омонимией. Так, модели GSD и Taiga, к примеру, не всегда
могли принять правильное решение о начальной форме аккузатива и генитива
существительного \emph{Господь} (\emph{Господа}), совпадающей с
номинативом множественного числа \emph{господин.} Лучшие результаты
показала модель \emph{SynTagRus}.

---\textgreater{} При распространении лемматизации на полный текст
Триоди необходимо выявить модель с наилучшим качеством. По результатам
экспериментов мы решили отказаться от PROIEL, а из современных моделей
выбирать между Taiga и SynTagRus, хотя их качество при лемматизации
службы одного дня оказалось не совсем идеальным.

\emph{\href{https://github.com/Drozhzhinastya/GSPC/tree/main/scripts/lemmatization}{Код}
и
\href{https://github.com/Drozhzhinastya/GSPC/tree/main/lemmatization/csl}{результаты}
лемматизации}

\hypertarget{ux433ux440ux435ux447ux435ux441ux43aux438ux439-ux442ux435ux43aux441ux442}{%
\section{Греческий
текст}\label{ux433ux440ux435ux447ux435ux441ux43aux438ux439-ux442ux435ux43aux441ux442}}

Для лемматизации древнегреческого текста используется модуль Backoff
Lemmatizer библиотеки CLTK. Выбор лемматизатора основан на выводах,
представленных в статье McGillivray-Vatri, авторы которой анализируют
точность четырех программных решений для лемматизации текста: Diorisis,
LAGT, GLEM, и CLTK. По результатам двух серий экспериментов CLTK показал
сравнительно высокую точность подбора корректной леммы, что и послужило
определяющим фактором при выборе данного модуля для работы в рамках
проекта.

Другим очевидным преимуществом \emph{backoff} модуля является
использование цепочки лемматизаторов: в случае неудачного результата при
обработке токена программа обращается ко внешним лемматизаторам, пока не
будет найдена лемма или не будут исчерпаны все указанные варианты.
Данный подход существенно повышает точность обработки текста, однако
имеет и определенные недостатки. Поскольку backoff-цикл прерывается при
первой найденной лемме, результаты лемматизации будут сильно отличаться
в зависимости от того порядка, в котором указаны внешние лемматизаторы в
backoff-цепочке.

В качестве экспериментальных корпусов в проекте McGillivray-Vatri
использовались классические тексты, что актуализирует проблему
идентификации языка текста Цветной Триоди.

Библиотека CLTK, как и прочие аналогичные инструменты, направлена на
обработку классических текстов, а модели Backoff Lemmatizer для
древнегреческого языка обучены на материалах проекта Perseus. Греческий
текст Цветной Триоди, обрабатываемый в рамках проекта, довольно сложно
поддается лингвистической классификации -- необходим более глубокий
анализ и консультация специалистов, чтобы выйти за рамки
сформулированного внутри проекта условного обозначения \emph{греческий
язык богослужебных текстов}. Как и в случае с условно церковнославянским
текстом, греческая Триодь находится на шкале между койне и греческим
Нового Завета, сочетая элементы обеих форм греческого языка.

Ниже представлены результаты лемматизации греческого фрагмента службы
Вознесения при помощи лемматизаторов CLTK и UDPipe, а также фрагмент
исходного текста для сравнения:

\begin{itemize}
\tightlist
\item
  Исходный текст:
\end{itemize}

\emph{Ὁ Κύριος ἀνελήφθη εἰς οὐρανούς, ἵνα πέμψῃ τὸν Παράκλητον τῷ κόσμῳ,
οἱ οὐρανοὶ ἡτοίμασαν τὸν θρόνον αὐτοῦ, νεφέλαι τὴν ἐπίβασιν αὐτοῦ,
Ἄγγελοι θαυμάζουσιν, ἄνθρωπον ὁρῶντες ὑπεράνω αὐτῶν, ὁ Πατὴρ ἐκδέχεται,
ὃν ἐν κόλποις ἔχει συναΐδιον. Τὸ Πνεῦμα τὸ ἅγιον κελεύει πᾶσι τοῖς
Ἀγγέλοις αὐτοῦ· Ἄρατε πύλας οἱ ἄρχοντες ἡμῶν, Πάντα τὰ ἔθνη κροτήσατε
χεῖρας. ὅτι ἀνέβη Χριστός, ὅπου ἦν τὸ πρότερον.}

\begin{itemize}
\tightlist
\item
  CLTK:
\end{itemize}

\emph{ὁ κύριος ἀναλαμβάνω εἰς οὐρανός ἵνα πέμπω ὁ παράκλητος ὁ κόσμῳ, ὁ
οὐρανός ἑτοιμάζω ὁ θρόνος αὐτός νεφέλη ὁ ἐπίβασις αὐτός ἄγγελος θαυμάζω
ἄνθρωπος ὁράω ὑπεράνω αὐτός ὁ πατήρ ἐκδέχομαι ὅς ἐν κόλπος ἔχω συναΐδιον
ὁ πνεῦμα ὁ ἅγιος κελεύω πᾶς ὁ ἄγγελος αὐτός ἀείρω πύλη ὁ ἄρχων ἡμεῖς πᾶς
ὁ ἔθνος κροτέω χείρ ὅτι ἀναβαίνω χριστός ὅπου εἰμί ὁ πρότερος}

\begin{itemize}
\tightlist
\item
  UDPipe - Perseus
\end{itemize}

\emph{ὁ Κύριος ἀαλαμβάνω εἰς οὐρανός, ἵνα πέμπω ὁ Παράκλητος ὁ κόσμος ,
ὁ οὐρανός ἑτοιμάζω ὁ θρόνος αὐτός , νεφέλα ὁ ἐπίβασις αὐτός , Ἄγγελοι
θαυμάζω , ἄνθρωπος ὁράω ὑπεράνω αὐτός , ὁ Πατὴρ ἐκδέχομαι , ὅς ἐν κόλπος
ἔχω συναΐδιος ὁ Πνεῦμα ὁ ἅγιος κελεύω πᾶς ὁ Ἀγγέλοί αὐτός · Ἄρατε πύλη ὁ
ἄρχων ἐγώ , πᾶς ὁ ἔθνος κροτάω χείρ ὅτι ἀνέβη Χριστός , ὅπου εἰμί ὁ
πρότερος}

Как видно из примеров, оба лемматизатора показывают довольно высокую
точность. CLTK не справился со словом \emph{ὁ κόσμος} (здесь в значении
«мир»), оставив его в форме дательного падежа. Возможной причиной ошибки
может являться \emph{iota subscriptum} (подстрочная йота),
встречающаяся, среди прочего, в формах существительных дательного
падежа. Потенциальным решением подобной проблемы может быть замена таких
букв на формы \emph{iota adscriptum} на этапе предварительной обработки
текста: ῳ --\textgreater{} ωι.

Фрагмент, обработанный при помощи UDPipe, содержит больше неточностей:
диалектная форма \emph{νεφέλα} вместо ожидаемого \emph{νεφέλη}
(«облако»), пропущенная буква «ню» в глаголе

\emph{ἀαλαμβάνω} (здесь «возношусь»), а также необработанные формы
\emph{ἄρατε} (\emph{αἴρω}, \emph{ἀείρω}, «поднимаю, беру») и
\emph{ἀνέβη} (\emph{ἀναβαίνω}, «восхожу»). При этом, UDPipe распознает
именованные сущности, что может оказаться актуальным, если будет решено
добавить подобный функционал в пайплайн.

\emph{\href{https://github.com/Drozhzhinastya/GSPC/tree/main/scripts/lemmatization}{Код}
и
\href{https://github.com/Drozhzhinastya/GSPC/tree/main/lemmatization/greek}{результаты}
лемматизации}

\bookmarksetup{startatroot}

\hypertarget{text-similarity}{%
\chapter{\texorpdfstring{\textbf{Text-similarity}}{Text-similarity}}\label{text-similarity}}

Задача оценки семантического сходства (text-similarity) заключается в
определении степени схожести двух предложений с точки зрения
транслируемого смысла.

Семантический поиск направлен на повышение точности сопоставления
текстовых единиц путем понимания содержания поискового запроса. В
отличие от использования только лексических совпадений, семантический
поиск также может находить синонимы и другие единицы, имеющие схожий
контекст употребления, поскольку этот алгоритм основывается на
эмбеддингах -- векторных представлениях текстовых единиц. Термин
``embedding'', взятый из английской литературы, используется для
описания процесса, когда модель оцифровывает смысл слов и записывает его
в виде упорядоченного набора числовых значений -- вектора.

Наиболее эффективным способом построения моделей естественного языка
является обучение нейронных сетей на основе архитектуры «трансформер». В
качестве примера можно привести
\href{https://arxiv.org/abs/1810.04805}{BERT} -- модель. которая
используется для определения сходства слов и предложений.

Более качественные результаты можно получить, оптимизируя BERT под
конкретные задачи. К примеру, модель
\href{https://arxiv.org/pdf/1908.10084.pdf}{SBERT} обучена
непосредственно для работы с задачами по определению схожести
предложений на основе косинусной меры.

Как отмечают разработчики
\href{https://habr.com/ru/companies/sberdevices/articles/527576/}{модели},
её ``архитектура представляет собой
\href{https://en.wikipedia.org/wiki/Siamese_neural_network}{сиамскую
нейронную сеть} с тремя входами для триплета~ «anchor --- positive ---
negative».~ К каждому из входов применяется модуль BERT, который и будет
выполнять роль NLU в этом эксперименте. Модуль содержит в себе
\href{https://paperswithcode.com/method/wordpiece}{wordpiece-токенизатор}
для преобразования входных строк в BERT-совместимый формат (input\_ids,
input\_mask, token\_type\_ids), а также саму обучаемую модель BERT для
векторизации текста.''

В результате дообучения модели SBERT для задачи поиска переводных
эквивалентов было получено множество вариантов
\href{https://huggingface.co/models?pipeline_tag=sentence-similarity\&sort=downloads\&search=multi}{мультиязычных
моделей}.

В настоящий момент мы считаем, что подобные мультиязыковые модели могут
значительно приблизить проект к решению изначальной задачи, потенциально
обеспечивая более высокое качество поиска наиболее схожих единиц
греческого и церковнославянского текста.

В репозитории представлен
\href{https://github.com/Drozhzhinastya/GSPC/blob/main/scripts/text-similarity/GSPC_sbert.ipynb}{код}
для работы с нашим текстом, а также результаты работы с моделью
sentence-transformers/all-MiniLM-L6-v2, которая на наш взгляд
демонстрирует более высокое качество сведения текстов Триоди.

Эксперименты проводились как на уровне отдельных предложений, так и на
уровне гимнов (минимальных единиц разметки текста, представленной в
нашей csv - см.выше). Каждое уникальное предложение/гимн греческого
текста сопоставлялось с каждым предложением церковнославянского текста.
Результаты экспериментов представлены в
\href{https://github.com/Drozhzhinastya/GSPC/tree/main/csv/sbert}{csv
таблице}, где собраны топ-5 церковнославянских предложений, выделенных
моделью как наиболее близкие соответствующему греческому элементу.

В дальнейшей перспективе предстоит разработать решение для оценки
качества сведения текстов и улучшения результатов выравнивания с опорой
на него.

\bookmarksetup{startatroot}

\hypertarget{ux432ux43eux43fux440ux43eux441-ux43eux431-ux43eux446ux435ux43dux43aux435-ux43aux430ux447ux435ux441ux442ux432ux430-ux441ux432ux435ux434ux435ux43dux438ux44f-ux442ux435ux43aux441ux442ux43eux432}{%
\chapter{\texorpdfstring{\textbf{Вопрос об оценке качества сведения
текстов}}{Вопрос об оценке качества сведения текстов}}\label{ux432ux43eux43fux440ux43eux441-ux43eux431-ux43eux446ux435ux43dux43aux435-ux43aux430ux447ux435ux441ux442ux432ux430-ux441ux432ux435ux434ux435ux43dux438ux44f-ux442ux435ux43aux441ux442ux43eux432}}

В этой связи на настоящий момент мы выявили 3 основных метода для оценки
качества результатов сведения потенциально эквивалентных единиц, каждый
из которых, тем не менее требует оптимизации:

\begin{itemize}
\tightlist
\item
  Качественный отсмотр результатов работы моделей и элайнеров
  специалистами (ресурсозатратно)
\item
  Создание золотого датасета / готового датасета выровненных текстов,
  который может быть использован для оценки качества выравнивания (в
  отличие от Цветной Триоди, результатов выравнивания которой в готовом
  виде нет). В качестве источника создания такого датасета могут быть
  использованы выровненные тексты
  \href{https://dhonorare.ru/texts/trebnik/molitvy-v-pervyy-den-posle-rozhdeniya-mladentsa}{DHonorare}
\item
  Создание золотого датасета + дообучение наиболее оптимальных моделей
  BERT на нём
\end{itemize}

\bookmarksetup{startatroot}

\hypertarget{ux43dux430ux43fux440ux430ux432ux43bux435ux43dux438ux44f-ux434ux430ux43bux44cux43dux435ux439ux448ux435ux433ux43e-ux438ux441ux441ux43bux435ux434ux43eux432ux430ux43dux438ux44f}{%
\chapter{\texorpdfstring{\textbf{Направления дальнейшего
исследования}}{Направления дальнейшего исследования}}\label{ux43dux430ux43fux440ux430ux432ux43bux435ux43dux438ux44f-ux434ux430ux43bux44cux43dux435ux439ux448ux435ux433ux43e-ux438ux441ux441ux43bux435ux434ux43eux432ux430ux43dux438ux44f}}

Не смотря на то, что изначальные исследовательские цели не
финализированы, работа над проектом проявила множество
особенностей/проблемных зон/вопросов, требующих дальнейшего уточнения.

Проект продолжается; мы продолжаем исследовать обозначенные методы и
инструменты, а также искать новые решения, которые могут приблизить нас
к искомому результату.

\begin{itemize}
\tightlist
\item
  оптимизация алгоритма проверки качества сведения текстов
\item
  выравнивание текстов Цветной триоди с оценкой качества
\item
  публикация параллельного корпуса на самостоятельной платформе или
  интеграция в \href{https://liturcorpora.ru/}{Liturcorpora}
\end{itemize}


\backmatter

\end{document}
